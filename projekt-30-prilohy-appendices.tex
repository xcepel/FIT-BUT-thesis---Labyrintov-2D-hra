% Tento soubor nahraďte vlastním souborem s přílohami (nadpisy níže jsou pouze pro příklad)

% Pro kompilaci po částech (viz projekt.tex), nutno odkomentovat a upravit
%\documentclass[../projekt.tex]{subfiles}
%\begin{document}

% Umístění obsahu paměťového média do příloh je vhodné konzultovat s vedoucím
%\chapter{Obsah přiloženého paměťového média}

%\chapter{Manuál}

%\chapter{Konfigurační soubor}

%\chapter{RelaxNG Schéma konfiguračního souboru}

%\chapter{Plakát}

\chapter{Struktura odevzdaného adresáře}\label{chap:file_directory}
\todo{txt s game assety}

\chapter{Výsledky experimentů s mapou}\label{chap:map_experiments}
\todo{tabulky s průměry}

\chapter{Výsledky uživatelského testování}\label{chap:user_testing}

\begin{table}[htbp]
\centering
\begin{tabularx}{\textwidth}{|X|X|X|}
\hline
\multicolumn{1}{|c|}{\textbf{otázka č. 1}} & \multicolumn{1}{c|}{\textbf{otázka č. 2}} & \multicolumn{1}{c|}{\textbf{otázka č. 3}} \\ \hline
\textbf{Kolik času jste strávili u hraní?} & \textbf{Jaká byla vaše nejvyšší dosažená úroveň (level)?} & \textbf{Jaký je váš celkový dojem ze hry?} (1 = velmi negativní - 10 = velmi pozitivní) \\ \hline
0:03:50 & 7 & 8 \\ \hline
0:07:00 & 12 & 7 \\ \hline
0:10:00 & 11 & 5 \\ \hline
0:10:55 & 9 & 8 \\ \hline
0:12:00 & 11 & 8 \\ \hline
0:23:00 & 10 & 8 \\ \hline
0:20:00 & 16 & 8 \\ \hline
0:20:00 & 8 & 9 \\ \hline
1:10:25 & 20 & 10 \\ \hline
0:20:00 & 11 & 9 \\ \hline
1:40:00 & 28 & 8 \\ \hline
0:30:00 & 20 & 7 \\ \hline
0:50:00 & 18 & 9 \\ \hline
0:30:00 & 11 & 5 \\ \hline
0:20:00 & 12 & 9 \\ \hline
0:20:00 & 6 & 9 \\ \hline
0:15:00 & 9 & 7 \\ \hline
0:15:00 & 15 & 10 \\ \hline
0:12:00 & 19 & 8 \\ \hline
0:40:00 & 22 & 7 \\ \hline
1:00:00 & 10 & 10 \\ \hline
0:58:25 & 15 & 9 \\ \hline
\end{tabularx}
\caption{1. část výsledků uživatelského hodnocení\,--\,obecné otázky ohledně hratelnosti.}
\end{table}

\begin{table}[htbp]
\centering
\begin{tabularx}{\textwidth}{|X|X|}
\hline
\multicolumn{1}{|c|}{\textbf{otázka č. 4}} & \multicolumn{1}{c|}{\textbf{otázka č. 5}} \\ \hline
\textbf{Jak byste ohodnotili uživatelské rozhraní a celkovou přehlednost hry?} (1 = velmi nepřehledné - 10 = velmi přehledné) & \textbf{Jak byste ohodnotili stabilitu a výkon hry (záseky/pády,...)?} \\ \hline
9 & Velmi stabilní a plynulá hra \\ \hline
9 & Velmi stabilní a plynulá hra \\ \hline
5 & Velmi stabilní a plynulá hra \\ \hline
8 & Velmi stabilní a plynulá hra \\ \hline
10 & Velmi stabilní a plynulá hra \\ \hline
10 & Velmi stabilní a plynulá hra \\ \hline
9 & Velmi stabilní a plynulá hra \\ \hline
10 & Velmi stabilní a plynulá hra \\ \hline
1 & Velmi stabilní a plynulá hra \\ \hline
8 & Velmi stabilní a plynulá hra \\ \hline
8 & Velmi stabilní a plynulá hra \\ \hline
8 & Velmi stabilní a plynulá hra \\ \hline
9 & Velmi stabilní a plynulá hra \\ \hline
7 & Středně stabilní, občasné pády a záseky \\ \hline
7 & Velmi stabilní a plynulá hra \\ \hline
7 & Velmi stabilní a plynulá hra \\ \hline
8 & Velmi stabilní a plynulá hra \\ \hline
10 & Velmi stabilní a plynulá hra \\ \hline
10 & Velmi stabilní a plynulá hra \\ \hline
4 & Stabilní hra, občas záseky \\ \hline
10 & Velmi stabilní a plynulá hra \\ \hline
9 & Velmi stabilní a plynulá hra \\ \hline
\end{tabularx}
\caption{2. část výsledků uživatelského hodnocení\,--\,Uživatelské rozhraní a stabilita}
\end{table}

\begin{table}[htbp]
\centering
\begin{tabularx}{\textwidth}{|X|X|X|}
\hline
\multicolumn{1}{|c|}{\textbf{otázka č. 6}} & \multicolumn{1}{c|}{\textbf{otázka č. 7}} & \multicolumn{1}{c|}{\textbf{otázka č. 8}} \\ \hline
\textbf{Co pro vás bylo nejtěžší?} & \textbf{Jak byste ohodnotil vyváženost zvyšující se obtížnosti?} & \textbf{Pokud jste hodnotili hru jako spíše nevyváženou, jaký k tomu byl důvod?} \\ \hline
Nepřátelé & 9 & \\ \hline
Nepřátelé & 9 & \\ \hline
Nepřátelé & 10 & \\ \hline
Nepřátelé & 8 & \\ \hline
Orientace v bludišti & 8 & \\ \hline
Orientace v bludišti & 9 & Ve vyšších levelech se mi spíše odečítaly body než přičítaly \\ \hline
Rychlost projití levelu & 9 & \\ \hline
Orientace v bludišti & 7 & Vlastně mi to přišlo fajn, ale ke konci už jsem bloudil opravdu dlouho v těch bludištích, takže už mě to ani nebavilo :( \\ \hline
Když člověk vleze do exitu z hora/dola a drží klávesu, tak se v menu trefí na exit hry. & 8 & U některých běhů nebyly skoro žádné předměty, u jiných jich bylo moc. \\ \hline
Orientace v bludišti & 4 & Nepocítila jsem rozdíly mezi jednotlivými úrovněmi, např. 1 a 3, 5 a 10. Nebyla tam jasná změna v obtížnosti, někdy stoupala příliš pomalu. \\ \hline
Orientace v bludišti & 9 & Všechno bylo vyvážené, dokud se tam neobjevil lučišník. \\ \hline
Orientace v bludišti & 10 & \\ \hline
Nepřátelé & 6 & Celková obtížnost se zvyšuje moc pomalu \\ \hline
Nepřátelé & 9 & \\ \hline
Nepřátelé & 10 & \\ \hline
Nepřátelé & 9 & \\ \hline
Nepřátelé & 8 & \\ \hline
Nepřátelé & 10 & \\ \hline
Nepřátelé & 8 & Celková obtížnost se zvyšuje moc pomalu \\ \hline
Orientace v bludišti & 8 & \\ \hline
Nepřátelé & 9 & \\ \hline
Nepřátelé & 10 & \\ \hline
\end{tabularx}
\caption{3. část výsledků uživatelského hodnocení\,--\,Náročnost a vyváženost}
\end{table}

\begin{table}[htbp]
\centering
\begin{tabularx}{\textwidth}{|X|X|X|}
\hline
\multicolumn{1}{|c|}{\textbf{otázka č. 9}} \\ \hline
\textbf{Jaké další funkce/vlastnosti/prvky byste ve hře ocenili/změnili?} \\ \hline

Legenda co dělají jednotlivé itemy a nepřátelé, vysvětlení jak se počítá skóre \\ \hline 

Opětovné sbírání mečů nemá nejspíše žádný další účinek na hraní. Což takhle \uv{stackovat} nebo zvyšovat damage? Předmět co zvyšuje rychlost by mohl vydržet déle než jenom těch cca 5 vteřin. \\ \hline

1. Na začiatku levelu by bolo dobré ukázať aké predmety je možné v tomto levely získať. \\ 
2. Odhalovacia minimapa.\\ 
3. Skóre počas levelu. Ja som dokonca získal záporné skóre. \\ \hline

Jen drobnost, ale pokud dojde k odečtení skóre, bylo by lepší skóre odečíst rovnou a nedávat +    -score. Bavím se nyní o obrazovce, která se zobrazí po dokončení levelu.\\ \hline

Celkovy cas straveny ve hre v danem levelu nebo v celkovem, lepsi viditelnosti zvoleneho pole v menu i pri death screen\\ \hline

Libí se mi variabilita nepřátel a bludišť, byla by fajn nějaká hudba, super by bylo si za body kustomizovat panáčka (čepičky a tak), hra mě fakt bavila, určitě ji budu hrát i dál \\ \hline

Změnil bych to, že při posunu nahoru je menší zorné pole než dolů.\\ \hline

ocenil bych kdyby šlo hru odpauznout klávesou escape, jo a btw, dvakrát během hraní se mi zapnul snipping tool (který obyčejně zapínám klávesou prtscr), možná (asi) je to nějaký issue u mě co se hrou nesouvisí, ale stalo se mi to dvakrát takže to je celkem sus, mám windows 11 23h2 build 22631.3374, ale jinak pecka moc se mi to líbí\\ \hline

Dobrá hra, pár věcí: Lze útočit přes rohy (pravý horní roh). Elixir zrychlení maže druhý elixir zrychlení. Ovládání ze hry se probublává do menu a tím se hra velmi snadno nechtěně vypne. Skore se nezobrazuje ve hře. Skoré se divně řadí. Občas je vzhůru nohama. Občas se přičítá negativní skóre? Meče se nesčítají/neobjevují v inventáři ve vícero instancích. Zpomalovací MOB je příliš těžký. Šlo by dodat vícero volných prostor, kde by se lépe bojovalo, než jen koridory? Poznačování prošlých úseků - třeba pomocí nějakého předmětu, který postava položí na zem? Past má když jsou schované ostany po levé straně podivné černé proužky. Nešel by trošku oddálit pohled? Když se běží nahoru, je kamera skoro u kraje hrací plochy a člověk snadno naběhne na ostny... Když se běží, tak postavička vibruje...\\ \hline

Není mi úplně jasná funkce některých (ochranných) prvků, např. štítu nebo bot.\\ \hline

pouze bych Přidal bych hudbu, animaci stromů(jak se hýbají ve větru) a nějaké \uv{consumable items} (předměty, které můžete vzít a aktivovat stisknutím tlačítka). Jinak je to skvělá hra.\\ \hline

Veľmi ma frustrovalo, že naraz na obrazovke vidím iba malú časť bludiska a dieliky sú obrovské. Páčil sa mi pixel art :)\\ \hline

Nevím, k čemu je sbírání truhel, jinak pěkná hra.\\ \hline

nepratele blokujici hrace jsou otravni; mozna trochu vice zvyraznit HP - nebylo mi jasne, jestli se jedna o zivoty co mam, nebo co muzu jeste doplnit + mi prijde, ze jsem tento placement neocekavala a dlouho jsem si jich vubec nevsimla - i s tim by vybarveni napr. na cerveno pomohlo\\ 
Ad. otazka pady - po dohrani a zadani jmena (bez diakritiky) a zmacknuti enteru hra spadla, po spusteni ale vse bylo ulozeno, to tak ciste pro info\\ 
Posledni poznamka: velke + za vizual, je to fakt pekny\\ \hline

\end{tabularx}
\caption{4. část výsledků uživatelského hodnocení\,--\,Návrhy testerů, část 1}
\end{table}


\begin{table}[htbp]
\centering
\begin{tabularx}{\textwidth}{|X|X|X|}
\hline
\multicolumn{1}{|c|}{\textbf{otázka č. 9}} \\ \hline
\textbf{Jaké další funkce/vlastnosti/prvky byste ve hře ocenili/změnili?} \\ \hline

maly vyhled je frustrujici, obzvlast v kombinaci s kolizemi s neprateli\\ \hline

- keď sa hráč pohybuje diagonálne, ide o dosť rýchlejšie\\ 
- nejaký zvuk by bol fajn, ale chápem, že to je iba alpha\\ 
- podľa animácie útoku som usúdil, že damage to dáva až ku koncu, ale potom som zistil, že ho dáva na začiatku\\ 
- tiles sú veľmi veľké\\ 
- páči sa mi ako hra vyzerá, má peknú farebnú paletu\\ 
- bigfoot ma zavrel snehuliakom do rohu a tam ma pomaly zabil, desivý týpek\\ \hline

Zrušil bych fullscreen po zapnutí, protože kvůli této \uv{feature} jsem nemohl přemístit hru na velkou obrazovku, musel jsem přesouvat video, které jsem měl puštěné do pozadí naopak na druhou obrazovku, což bylo zbytečně komplikované, protože hra byla ve fullscreenu nad videem.\\ 
Nepřátelům se obvykle nedá vyhnout kvůli úzkým koridorům, ocenil bych aby bylo reálně možné se jim vyhýbat.\\ 
Zpomalení od slima je otravně velké, snížil bych ho.\\ 
Záporný počet bodů za relativně rychle projitý level (oproti několika předchozím kde jsem byl pomalejší) mě celkem zarazil, zamyslel bych se nad způsobem výpočtu bodů.\\ 
Chybí vysvětlení že je možné nějak útočit.\\ 
Po skončení levelu se snadno překlikne v důsledku držení šipky ze hry na menu/vypnutí hry a dojde k nechtěnému vypnutí, ocenil bych otázku, zda si je volbou uživatel jistý nebo jinou formu, aby nedošlo k neúmyslnému vypnutí.\\ \hline

Hra se mi moc líbí, design je jednoduchý ale tak krásný, jednou chybu kterou jsem tam já osobně měla (mohla to být chyba v mojí nepozornosti), v jedné úrovni jsem nemohla najít portál do dalšího levelu a opravdu jsem to tam prošla celé, ale to je tak jediný negativní postřeh který jsem zaznamenala. Jinak ej to moc super! :-))\\ \hline

Rychlejší pohyb postavy skrz item bot \\ \hline

Přidání legendy nebo označení, co některé věci dělají. Například truhla by mohla mít nějaké zaznačení přidání bodů do skóre.\\ 
Přidání počitadla skóre do hry nejen při dokončení levelu.\\ 
Po sebrání meče se mod nepřidá - není jasné zda se poškození zvýší nebo že není možné mít více zbraní zároveň. \\ 
Útok divočáka skončil chybou, po zabití se divočák přichytil k hratelné postavě a po krátký časový interval v postavě byly zaseklé jeho ostatky.\\ 
Po ukončení hry a opětovném načtení hra vytvořila na levelu 21 mapu pro level 1.\\ \hline

někdy mi z truhličky nepadaly odměny, nevím jestli se jedná o bug nebo co by měla dělat, nepochopil jsem\\ \hline

Při hře mít  viditelné stopky, když se sebere meč podruhé - třeba nějaký upgrade, po skončení levelu by bylo hezké, kdyby se ukázala mapa daného labyrintu \\ \hline

\end{tabularx}
\caption{4. část výsledků uživatelského hodnocení\,--\,Návrhy testerů, část 2}
\end{table}


% Pro kompilaci po částech (viz projekt.tex) nutno odkomentovat
%\end{document}
